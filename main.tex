\documentclass[a4paper,dvipsnames, 11pt]{amsart}
\usepackage{preamble}
\usepackage{lipsum}
\vspace{10ex}

\begin{document}
\maketitle
\cite{Cis19}
\begin{notation}
	We employ the following notations.
	\begin{itemize}
		\item %
			Possibly large (2,1)-categories are denoted by bf symbols:
			$\C,\one{A},\one{E},\ldots$
		\item %
			Small categories are denoted by bb symbols:
			$\dgm{I},\dgm{D},\dgm{A},\ldots$
		\item %
			$\Set$ is the category of small sets.
		\item %
			$\Catbi$ is the 2-category of small categories.
		\item %
			$\Cat$ is the (2,1)-category of small categories.
		\item %
			$\ra$ is the $1$-simplex seen as a category.
			$\C^\ra$ is the arrow category of $\C$.
		\item %
			$\C_{/A}$ and $\C_{A/}$ are over and under categories respectively.
		\item %
			$\cod$ and $\dom$ means codomain and domain respectively. They often have as their type
			$\C^\ra\arr\C$, $\C_{/A}\arr\C$,
			or $\C_{A/}\arr\C$.
		\item %
			By a \emph{replete class of morphisms} of $\C$, we mean a replete subcategory of $\C^\ra$ that is a groupoid.
		\item %
		\qedhere %
	\end{itemize}
\end{notation}
\begin{definition}
	\[
		\begin{tikzcd}
			\StrCat
			\ar[r]
			\ar[d]
				&
				\Cat^\ra
				\ar[d,"\cod"]
			\\
			\Set
			\ar[r]
				&
				\Cat
		\end{tikzcd}
	\]
\end{definition}
\begin{definition}
	A \emph{clan} $\C=(\C,\Fib)$ is a pair of a category $\C$ and a replete class $\Fib$ of morphisms satisfying the following conditions.
	Arrows in $\Fib$ are called \emph{fibrations} of $\C$.
	\begin{itemize}
		\item %
			$\C$ has a terminal object.
		\item %
			Let $h\colon A\arr B$ and $f\colon X\arr[fib] B$ be morphisms in $\C$ such that $f$ is a fibration.
			Then there is a pullback square
			\[
				\begin{tikzcd}
					\cdot
					\ar[r]
					\ar[d,fib]
					\pullback[rd]
						&
						X
						\ar[d,"f",fib]
					\\
					A
					\ar[r,"h"']
						&
						B
				\end{tikzcd}
			\]
			in $\C$
			such that the left side is also a fibration.
		\item %
			For each object $A\in\C$, the unique morphism $A\arr 1$ towards the terminal object
			is a fibraion.
		\item %
			$\Fib$ is closed under composition.
		\qedhere %
	\end{itemize}
\end{definition}
\newpage

\begin{theorem}
	shoumei no aidani claim ireru yatu
\end{theorem}
\begin{proof}
	\lipsum[1]
	\begin{claim}
		nannka claim siro
	\end{claim}
	\begin{since}
		\lipsum[2]
	\end{since}
	\lipsum[3]
\end{proof}
\begin{theorem}
	Saigo ni claim kuru taipu.
\end{theorem}
\begin{proof}
	\lipsum[1]
	This follows from the following Claim, which completes the proof.
	\qedhere
	\begin{claim}
		nannka claim siro
	\end{claim}
	\begin{since}
		\lipsum[2]
	\end{since}
\end{proof}

\bibliographystyle{halpha-abbrv}
\bibliography{bibliography}

\end{document}
