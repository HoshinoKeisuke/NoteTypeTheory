\documentclass[a4paper,dvipsnames, 11pt]{amsart}
\usepackage{preamble}
\usepackage{lipsum}
\vspace{10ex}

\begin{document}
\maketitle
\cite{Cis19}
\begin{notation}
	We employ the following notations.
	\begin{itemize}
		\item %
			(2,1)-categories are denoted by bf symbols:
			$\C,\one{A},\one{E},\ldots$
			or bb symbols:
			$\dgm{I},\dgm{D},\dgm{A},\ldots$
		\item %
			$\Set$ is the category of sets.
		\item %
			$\Cat$ is the large (2,1)-category of categories.
		\item %
			There is a fully faithful (2,1)-functor $\disc\colon\Set\arr[hook]\Cat$.
		\item %
			$\Catbi$ is the large 2-category of categories.
		\item %
			$\ra$ is the $1$-simplex seen as a category.
			$\C^\ra$ is the arrow category of $\C$.
		\item %
			$\C_{/A}$ and $\C_{A/}$ are over and under categories respectively.
		\item %
			$\cod$ and $\dom$ mean codomain and domain respectively. They often have as their type
			$\C^\ra\arr\C$, $\C_{/A}\arr\C$,
			or $\C_{A/}\arr\C$.
		\item %
			By a \emph{replete class of morphisms} of $\C$, we mean a replete full subcategory of $\C^\ra$.
		\item %
			$\WSubone$ is the replete full sub (2,1)-category of $\Cat^\ra$
			consisting of wide subcategory inclusions; i.e., faithful functors that induce equivalences on core groupoids.
		\item %
		\qedhere %
	\end{itemize}
\end{notation}
\section{Dependent type theories in terms of display maps}
\begin{definition}
	A \emph{display map category} $\C=(\C,\dis_\C)$ is a pair of a category $\C$ and a replete class $\dis_\C$ of morphisms satisfying the following conditions.
	Arrows in $\dis_\C$ are called \emph{display maps} of $\C$.
	\begin{itemize}
		\item %
			$\C$ has a terminal object.
		\item %
			Let $h\colon \Delta\arr \Gamma$ and $f\colon A\arr[fib] \Gamma$ be morphisms in $\C$ such that $f$ is a display map.
			Then there is a pullback square
			\begin{equation}
				\label{dgm:substPullback}
				\begin{tikzcd}
					\cdot
					\ar[r]
					\ar[d,fib]
					\pullback[rd]
						&
						A
						\ar[d,"f",fib]
					\\
					\Delta
					\ar[r,"h"']
						&
						\Gamma
				\end{tikzcd}
			\end{equation}
			in $\C$
			such that the left side is also a display map.
		\qedhere %
	\end{itemize}
\end{definition}
\begin{definition}
	Define the 2-category $\CwD$ of display map categories
	as follows.
	\begin{itemize}
		\item %
			0-cells are display map categories.
		\item %
			1-cells are functors preserving display maps and pullbacks of the form \Cref{dgm:substPullback}.
		\item %
			A 2-cell $\alpha\colon F\arr[Rightarrow]G$ is a natural transformation such that
			naturality squares at display maps are pullback squares.
		\qedhere %
	\end{itemize}
\end{definition}
\begin{definition}
	A display map category is \emph{democratic} if for each object $\Gamma\in\one{C}$,
	there exists a sequence of display maps from $\Gamma$ to the terminal object.
	We write $\CwDdm$ for the full sub 2-category of $\CwD$ spanned by democratic display map categories.
\end{definition}
\begin{proposition}
	$\CwDdm$ is a (2,1)-category.
\end{proposition}
\begin{definition}
	A \emph{clan} $\C=(\C,\fib_\C)$ is a display map category satisfying the following conditions.
	Display maps (i.e., arrows in $\fib_\C$) are called \emph{fibrations} of $\C$.
	\begin{itemize}
		\item %
			For each object $A\in\C$, the unique morphism $A\arr 1$ towards the terminal object
			is a fibraion.
		\item %
			$\fib_\C$ is closed under composition.
		\qedhere %
	\end{itemize}
	We write $\Clan$ for the full sub 2-category of $\CwD$ spanned by clans. Since clans are always democratic, this is a (2,1)-category.
\end{definition}
\begin{definition}
	A \emph{strict category} is a category equipped with its wide subcategory that is a set.
	In other words, the (2,1)-category $\StrCat$ of strict categories is defined by the following pullback square.
	\[
		\begin{tikzcd}
			\StrCat
			\ar[r]
			\ar[dd,"\Obj"']
			\pullback[rd]
				&
				\WSubone
				\ar[d,hook]
			\\
				&
				\Cat^\ra
				\ar[d,"\dom"]
			\\
			\Set
			\ar[r,"\disc"']
				&
				\Cat
		\end{tikzcd}
	\]
	The functor on the left is locally fully faithful and surjective on objects.
	In particular, $\StrCat$ is a category.
\end{definition}
\begin{definition}
	A \emph{contextual category} $\C=(\C,\dis_\C,\emptycon,\ell)$ consists of the following data.
	\begin{itemize}
		\item %
			A strict category $\C$.
		\item %
			A structure $\dis_\C$ of a display map category for the underlying category of $\C$.
		\item %
			An object $\emptycon$ of $\C$ which is a terminal object in the underlying category of $\C$.
		\item %
			A function $\ell\colon\Obj(\C)\arr\dN$ satisfying the following conditions.
			\begin{itemize}
				\item %
					$\ell^{-1}(0)=\{\emptycon\}$.
				\item %
					The canonical function
					\[
						(\Gamma,A,f)\mapsto A\colon\coprod_{\Gamma\in\ell^{-1}(n)}\{(A,f)\,|\,f\colon A \arr[fib]\Gamma\}\arr\Obj(\C)
					\]
					is a monomorphism and
					$\ell^{-1}(n+1)$ coincides with its image.
			\end{itemize}
	\end{itemize}
	We write $\CwC$ for the category of contextual categories and functors preserving those structures.
\end{definition}
\begin{theorem}
	There exists an 2-adjunction
	\[
		\adjunction{\CwC}{\CwD}{\abs{-}}{\ctx}
	\]
	that restricts to a biequivalence
	\[
		\adjunction(\simeq){\CwC}{\CwDdm}{\abs{-}}{\ctx}
	\]
\end{theorem}
\section{Uemura's logical framework}
\begin{definition}
	A \emph{representable map category} $\R=(\R,\rep_\R)$ is a display map category satisfying the following conditions.
	Display maps (i.e., arrows in $\rep_\R$) are called \emph{representable maps} of $\R$.
	\begin{itemize}
		\item %
			$\R$ is finitely complete.
		\item %
			For each representable map $f\colon X\arr[fib]Y$,
			the pullback functor $f^*\colon\R_{/Y}\arr\R_{/X}$ has a right adjoint $f_*$.
		\qedhere %
	\end{itemize}
\end{definition}
\newpage

\begin{theorem}
	shoumei no aidani claim ireru yatu
\end{theorem}
\begin{proof}
	\lipsum[1]
	\begin{claim}
		nannka claim siro
	\end{claim}
	\begin{since}
		\lipsum[2]
	\end{since}
	\lipsum[3]
\end{proof}
\begin{theorem}
	Saigo ni claim kuru taipu.
\end{theorem}
\begin{proof}
	\lipsum[1]
	This follows from the following Claim, which completes the proof.
	\qedhere
	\begin{claim}
		nannka claim siro
	\end{claim}
	\begin{since}
		\lipsum[2]
	\end{since}
\end{proof}

\bibliographystyle{halpha-abbrv}
\bibliography{bibliography}

\end{document}
